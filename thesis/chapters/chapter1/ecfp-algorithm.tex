One widely used solution for molecular representation is Extended Connectivity Fingerprints (ECFP), a~class of~topological fingerprints based on neighbourhood information of~atoms~\cite{rogers2010ecfp}. They are intended to~reflect the~specific substructures present in the molecular graph. By~adjusting which features are included in~the~generic algorithm, they can be adapted to~encode specific properties of~a compound. The~results can be tuned by choosing what to~consider when assigning the~initial identifiers: atomic type and mass, bond multiplicities to~non-hydrogen neighbours, etc.

\subsection{The ECFP algorithm}
The ECFP encoding is computed with the~following algorithm \cite{rogers2010ecfp}:
\begin{algorithm}[H]
    \caption{ECFP fingerprint generation}
    \label{alg:ecfp}
    \begin{algorithmic}[1]
    \Require{Graph \( G = (V, E) \)}
    \Ensure{Set of identifiers \( L \)}
        \For{\( v_j \in V \)}
        \State \( h_j^0 \gets \text{id}(v_j) \)
        \EndFor
        \State \( L \gets \set{h_j^0 \colon j \in \abs{V}} \)
        \For{\( i \in \set{1, \dots, r} \)}
        \State \( L_j^i \gets \text{sort}\!\pars{\set{\pars{e_{jk}, h_k^{i-1}} \colon v_jv_k \in E}} \), where \( e_{jk} \) is the~multiplicity of~bond \( (v_j, v_k) \)
        \State \( h_j^i \gets \text{hash}\!\pars{i, \ h_j^{i-1}, \ L_j^i} \)
        \State \( L.\text{insert}(h_j^i) \)
        \EndFor
        \State \( L \gets \text{unique}(L) \)
        \State \Return \( L \)
    \end{algorithmic}
\end{algorithm}

As a standard approach, the~generated list of identifiers, also called connectivity values, is~transformed into a binary vector of a fixed length \( N \) \cite{leach2007chemoinformatics}. The most commonly used parameter values are \( N \in \set{1024, 2048, 4096} \) and \( r \in \set{1, 2, 3} \) \cite{landrum2012fingerprints}. The final representation in~the~form of~a~binary vector can be interpreted as a dictionary indexed by encoded substructures.

As follows from the~algorithm, the~properties of~ECFP encoding strongly depend on the~choice of~parameters, in~particular the~value of~\( r \). The~parameter \( r \) determines the~number of~generated identifiers, which is bounded from above by \( (r+1) \) times the~number of~atoms in~the~molecule. The~algorithm may return fewer identifiers if certain substructures are repeated or hash collisions occur. Thus, at some level, the~ratio of~the number of~generated identifiers to~the~number of~atoms can provide a~measure of~the complexity of~the structure. If a~molecule contains various substructures, it is sufficient to~consider the~close neighbourhood of~atoms to~distinguish it from a~similar compound. The~choice of~\( r \) is crucial for determining similarity in~the~Jaccard metric. It is possible to~give an example of~molecules that need the~value of~parameter \( r \) linear in~terms of~their size to~distinguish between them. As an example, two carbon chains of~lengths \( k \) and \( k+1 \) for any \( r \) less than \( \frac{k-1}{2} \) will be considered identical under the~Jaccard metric (see Figure~\ref{fig:chain}).

\begin{figure}[H]
\centering
\resizebox{0.85\textwidth}{!}{
    \begin{circuitikz}
\tikzstyle{every node}=[font=\large]
    \draw [line width=0.2mm,  short] (0,8.5) -- (1.5,10);
    \draw [line width=0.2mm,  short] (1.5,10) -- (3,8.5);
    \draw [line width=0.2mm,  short] (3,8.5) -- (4.5,10);
    \draw [line width=0.2mm,  short] (4.5,10) -- (6,8.5);
    \draw [line width=0.2mm,  short] (6,8.5) -- (7.5,10);
    \draw [line width=0.2mm,  short] (7.5,10) -- (9,8.5);
    \node at (0,8) {0};
    \node at (1.5,10.5) {1};
    \node at (3,8) {2};
    \node at (4.5,10.5) {3};
    \node at (7.5,10.5) {\dots};
    \node at (9,8) {\( k \)};
    
    \draw [line width=0.2mm,  short] (12,8.5) -- (13.5,10);
    \draw [line width=0.2mm,  short] (13.5,10) -- (15,8.5);
    \draw [line width=0.2mm,  short] (15,8.5) -- (16.5,10);
    \draw [line width=0.2mm,  short] (16.5,10) -- (18,8.5);
    \draw [line width=0.2mm,  short] (18,8.5) -- (19.5,10);
    \draw [line width=0.2mm,  short] (19.5,10) -- (21,8.5);
    \draw [line width=0.2mm, color={figBlue}, short](22.5,10) -- (21,8.5);
    \node at (12,8) {0};
    \node at (13.5,10.5) {1};
    \node at (15,8) {2};
    \node at (16.5,10.5) {3};
    \node at (19.5,10.5) {\dots};
    \node at (21,8) {\( k \)};
    \node at (22.5,10.5) {\( k+1 \)};
\end{circuitikz}
}
\caption{Carbon chains \( P_{k} \) and \( P_{k+1} \)}
\label{fig:chain}
\end{figure}

The ECFP algorithm is sensitive to~local changes, but not to~global ones. Replacing a~certain substructure with another that was not previously present significantly affects the~result returned by the~algorithm, whereas swapping substructures within a~compound does not. For instance, the~structure of~a carbon chain with substituents can be thought of~as a~tree, where the~substituents are subtrees. If we swap two subtrees, the~result is non-isomorphic to~the~original tree, but for small \( r \) the~encoding will not change, meaning that in~the~Jaccard metric the~trees will be identical. As an example, the~trees shown in~Figure~\ref{fig:nonisomorphic} for \( k = 10 \) and \( l = 7 \) have a~similarity of~\( 1.0 \) when the~fingerprints are generated using the~ECFP algorithm with parameters \( r=2 \) and \( N \geq 2048 \).

\begin{figure}[H]
\centering
\resizebox{0.7\textwidth}{!}{
    \begin{circuitikz}
\tikzstyle{every node}=[font=\Huge]
    \draw [line width=0.4mm,  short] (2,3.25) -- (10,15);
    \draw [line width=0.4mm,  short] (17,5.25) -- (10,15);
    \draw [line width=0.4mm,  short] (10,9.25) -- (10,15);

    \draw [line width=0.4mm,  short] (0,0) -- (4.25,0);
    \draw [line width=0.4mm,  short] (0,0) -- (2,3.25);
    \draw [line width=0.4mm,  short] (2,3.25) -- (4.25,0);
    \node at (2,1) {\( T_1 \)};
    
    \draw [line width=0.4mm,  short] (15,2) -- (19.25,2);
    \draw [line width=0.4mm,  short] (17,5.25) -- (15,2);
    \draw [line width=0.4mm,  short] (17,5.25) -- (19.25,2);
    \node at (17,3) {\( T_3 \)};
    
    \draw [line width=0.4mm,  short] (8,6) -- (12.25,6);
    \draw [line width=0.4mm,  short] (10,9.25) -- (8,6);
    \draw [line width=0.4mm,  short] (12.25,6) -- (10,9.25);
    \node at (10,7.2) {\( T_2 \)};
    

    \draw [line width=0.4mm,  short] (27,3.25) -- (35,15);
    \draw [line width=0.4mm,  short] (42,5.25) -- (35,15);
    \draw [line width=0.4mm,  short] (35,9.25) -- (35,15);

    \draw [line width=0.4mm,  short] (25,0) -- (29.25,0);
    \draw [line width=0.4mm,  short] (25,0) -- (27,3.25);
    \draw [line width=0.4mm,  short] (27,3.25) -- (29.25,0);
    \node at (27,1) {\( T_3 \)};
    
    \draw [line width=0.4mm,  short] (40,2) -- (44.25,2);
    \draw [line width=0.4mm,  short] (42,5.25) -- (40,2);
    \draw [line width=0.4mm,  short] (42,5.25) -- (44.25,2);
    \node at (42,3) {\( T_1 \)};
    
    \draw [line width=0.4mm,  short] (33,6) -- (37.25,6);
    \draw [line width=0.4mm,  short] (35,9.25) -- (33,6);
    \draw [line width=0.4mm,  short] (37.25,6) -- (35,9.25);
    \node at (35,7.2) {\( T_2 \)};

    \node at (4,8.5) {\( k \)};
    \node at (29,8.5) {\( k \)};
    \node at (16,8.5) {\( l \)};
    \node at (41,8.5) {\( l \)};
\end{circuitikz}
}
\caption{Non-isomorphic trees corresponding to~hydrocarbons with swapped substituents}
\label{fig:nonisomorphic}
\end{figure}
Overall, the~choice of~the parameters \( r \) and \( N \) affects both the~density of~fingerprints and the~results of~the similarity analysis of~encoded structures.