Combinatorics for enumerating isomers of~a~compound, topology for the purpose of studying entanglements of~polymer strands or graph theory for sampling a~molecule from chemical space - those are examples of~topics at the~intersection of~mathematics and computer science that have found significant applications in~cheminformatics \cite{1995mathematical}. They provide fundamental methods for \emph{in~silico} analysis of~chemical compounds \cite{1995mathematical}. With its development, the~efficient digital representation of~molecular structures is becoming crucial. It is particularly important as one of~the key features of~cheminformatics methods is their focus on calculations involving large collections of~small molecules \cite{leach2007chemoinformatics}. It is also relevant in~the~context of~the vast size of~the chemical space. The~estimated number of~potential compounds with drug-like properties is about \( 10^{60} \), a~number impossible to store physically as samples \cite{leach2007chemoinformatics}. It is therefore necessary to represent and store a~digital form of~at least some of~the chemical structures efficiently because the~entire collection is still far from accessible.

Some of~the largest databases of~chemical structures are Enamine REAL containing 9.6 billion compounds, ZINC20 with 750 million compounds, or PubChem with 112 million compounds \cite{kim2025pubchem, 2025enamine, 2025zinc}. One of~their main applications is virtual screening, a~procedure for scoring, and filtering compounds based on similarity and substructure search or prediction of~properties \cite{leach2007chemoinformatics}. Virtual screening aims to rank a~set of~chemical structures, allowing, for example, to identify compounds similar to those that exhibit the~desired properties or select those for synthesis and further analysis \cite{brown2015medicinal}. Digital databases offer the~possibility of~storing ``virtual molecules'', which are structures that have not yet been obtained, but can be synthesised with modern methods and are potentially of~interest in~screening \cite{leach2007chemoinformatics}.

A popular format for representing chemical compounds is the~SMILES (Simplified Molecular Input Line Entry System) notation, decoding the~structure of~a~compound into a~sequence of~letters (atomic symbols), numbers and special characters describing bond types, and brackets to indicate nested substructures \cite{brown2015medicinal, 2025smiles}. This notation allows compounds to be stored as strings of~different lengths, each in~a~separate line. The~key property is that it can be converted from and to a~graph representation without loss of~information about connectivity. Although SMILES is a~convenient and commonly used method for storing data on a~computer, it is typically converted back into a~molecular graph for computational purposes \cite{leach2007chemoinformatics}.

A natural approach to representing molecules is therefore to consider them as graphs and then encode them in~numerical form using graph methods. This corresponds well to one of~the main questions in~cheminformatics, namely whether two compounds are similar and whether a~compound contains a~given substructure, for example, a~functional group responsible for certain property or a~core structure of~a~class of~compounds \cite{karthikeyan2014chemoinformatics}. Determining the~common substructures of~molecular graphs reduces to the~subgraph isomorphism problem, which is NP-complete \cite{garey1990intractability}. Although heuristic algorithms such as Simulated Annealing and Genetic Algorithm have been developed for this problem, their application to ultra-large chemical databases remains inefficient \cite{leach2007chemoinformatics, li2016heuristics}. The~procedure used in~practice applies a~two-step search, first eliminating non-matching molecules from further screening, and then evaluating only the~potential candidates \cite{leach2007chemoinformatics}. The~method is based on performing comparisons and calculations on numerical representations instead of~molecular graphs, thus approaching the~difficult problem of~graph isomorphism by simpler vector comparisons. It includes the~heuristic assumption that such a~simplified representation preserves sufficient information about mutual similarity and substructures present in~the~molecular graphs.

Different types of~compound encodings can be classified by their type and the~complexity of~the information they capture. One-dimensional descriptors are single numerical values derived from a~molecule’s structure or properties \cite{leach2007chemoinformatics}. Examples include polar surface area, molecular weight or the~number of~hydrogen bond acceptors and donors. Some of~1D descriptors can only be determined experimentally \cite{brown2015medicinal}. Two-dimensional descriptors, often referred to as fingerprints, are binary or numerical vectors derived from the~molecular graph, representing the~molecule’s connectivity \cite{leach2007chemoinformatics}. The~most extensive, 3D descriptors are based on the~full three-dimensional arrangement of~a~molecule - the~coordinates of~individual atoms or angles between bonds. They can be extended to include also physico-chemical properties of~a~compound \cite{gasteiger2003chemoinformatics}. The~descriptors can capture features of~the entire structure or specific functional groups. The~choice of~a~suitable numerical representation is crucial, for example, for initial filtering of~non-matching molecules in~database queries. The~selected descriptor should be adjusted to the~analysed collection of~compounds. A~set of~features may effectively distinguish a~range of~organic compounds, but it might be insufficient when differentiating molecules within a~single family, such as hydrocarbons \cite{leach2007chemoinformatics}.

For similarity and substructure searches, 2D descriptors are most commonly used, as they offer a~balance between the~accuracy of~information and the~efficiency of~storage and calculations. The~main types of~2D chemical fingerprints are structural keys and hashed fingerprints \cite{leach2007chemoinformatics}. Structural keys, such as MACCS (Molecular ACCess System), are generated based on a~predefined dictionary of~molecular fragments, where each bit represents the~presence of~a~specific substructure \cite{leach2007chemoinformatics}. The~dictionary must be carefully adapted to the~dataset, as adding new substructures would require updating all fingerprints stored in~the~database. Nevertheless, the~advantage of~this type of~encoding is its ease of~interpretation. The~second type is hashed fingerprints, such as ECFP (Extended-Connectivity Fingerprint), consisting of~hash values of~molecule substructures projected onto a~vector \cite{leach2007chemoinformatics}. The~hash of~each substructure of~a~newly added compound is included in~the~encoding, without requiring changes to existing fingerprints.

Among various types of~molecular representations, Extended-Connectivity Fingerprints have become prominent \cite{karthikeyan2014chemoinformatics, leach2007chemoinformatics}. As binary fingerprints, they are especially well-suited for substructure search, as a~quick comparison of~bits can immediately reveal the~absence of~the query substructure (when not all the~corresponding bits are set) \cite{leach2007chemoinformatics}. Another noteworthy practical advantage is that fingerprints like ECFP can serve as a~form of~encryption for chemical structures. In collaboration between different companies, where structural data is exchanged, it is often desirable to keep the~exact structures private. In such cases, one may only share the~ECFP encodings which are sufficient for structure-activity relationship analysis but are not easily reversible to unambiguously reconstruct the~original compounds \cite{le2020decipher}. The~ECFP representation also has properties that make it particularly well-suited for use with the~Jaccard similarity (also known as Tanimoto similarity), which is favoured due to its fast computation and natural interpretation for binary vectors \cite{karthikeyan2014chemoinformatics}.

The aim of~this study is to analyse the~properties of~the ECFP algorithm and, on this basis, to evaluate and compare dimensionality reduction methods applied to ECFP fingerprints, in~the~context of~efficient similarity search.