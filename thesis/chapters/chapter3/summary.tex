The study has shown that both Johnson–Lindenstrauss and MinHash reduction methods provide an effective means of compressing ECFP binary fingerprints while preserving pairwise similarity. In addition to their solid theoretical foundations, the methods offer several practical benefits. Experiments on real-world chemical compound datasets confirmed that the false positive rate in KNN-search using encodings reduced with MinHash does not exceed 5\% at a fixed accuracy threshold of 0.1. This suggests that the evaluated reduction methods may also perform well in~clustering applications, for instance, in chemical space partitioning. It would also be valuable to assess how well these approaches perform in predictive modelling tasks, such as QSAR or~QSPR.

A remarkable advantage of the MinHash method is the possibility to compress ECFP fingerprints with a larger radius (a higher value of parameter \( r \)) without increasing collision rates. This allows more structural information to be captured while maintaining compression efficiency and accuracy in distinguishing similar molecules.

From an implementation perspective, both the Johnson–Lindenstrauss and MinHash methods are highly parallelizable, as each coordinate of the reduced fingerprint is computed independently. Given that the original fingerprint length is on the order of \( 10^3 \) and the reduced length is \( 10^2 \), it is possible to transform a single vector using GPU in nearly constant time. However, a~technical drawback is that the resulting vectors have non-integer values. In the case of Johnson-Lindenstrauss projection, this problem can be solved by sampling the projection matrix from a~discrete distribution with suitable mean and variance, such as the Rademacher distribution over \( \set{-1, 1} \), and for the MinHash method, the hash functions would require adaptation.

Building on these promising results, several questions remain to be explored. The main challenge lies in improving the selection of parameters to best preserve similarity information between encoded molecular structures. By studying how the size of molecules and diversity of functional-groups influence the rate of collisions of identifiers and the minimal value of parameter \( r \) required to distinguish molecules, it may be possible to optimize the encoding method and the fingerprint length. Adaptive reduction method of encodings based on an entire dataset is a particularly interesting topic that was not deeply investigated in this work but holds potential for improving practical results in applications such as KNN-search or clustering. As indicated by differences in~results for a random set of small molecules and a family of hydrocarbons, adjusting the encoding strategy to a specific dataset is important. Rather than applying uniform compression to all fingerprint bits, it may be beneficial to identify and merge highly correlated identifiers, or discard bits that are present in the vast majority of compounds and thus are not selective.

The studied methods, while already effective, leave room for further refinement, adapting them to dataset characteristics. As such, Johnson-Lindenstrauss and MinHash reductions represent a~starting point for exploration in efficient molecular representation.